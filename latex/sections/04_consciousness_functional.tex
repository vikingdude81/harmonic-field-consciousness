\section{A Consciousness Functional Based on Harmonic Field Dynamics}
\label{sec:consciousness-functional}

We now introduce the central construct of this paper: a scalar functional $C(t)$ that quantifies the degree of consciousness supported by a given harmonic field configuration. This functional is designed to capture key features associated with conscious states---complexity, integration, coherence, dynamical vitality, and proximity to criticality---while remaining grounded in the mathematically precise framework of the preceding sections.

\subsection{Motivation and Desiderata}

Before specifying the functional, we articulate the properties it should possess:

\begin{enumerate}
\item \textbf{Sensitivity to mode distribution.} Consciousness should depend not on the total power in the system but on how that power is distributed across modes. A state dominated by a single mode should have lower $C(t)$ than a state with energy spread across many modes.

\item \textbf{Integration and differentiation.} Following integrated information theory \cite{tononi2004information, oizumi2014phenomenology} and related frameworks, conscious states should exhibit both differentiation (many distinguishable configurations) and integration (global coherence). Our functional should capture both aspects.

\item \textbf{Dynamical non-equilibrium.} Conscious brains are metabolically active, dissipative systems far from thermodynamic equilibrium \cite{friston2010free}. The functional should reflect this through a measure of entropy production.

\item \textbf{Criticality.} Empirical evidence links consciousness to near-critical dynamics \cite{tagliazucchi2012criticality, carhart2014entropic}. The functional should reward states near the edge of chaos.

\item \textbf{Frequency-band agnosticism.} The functional should not privilege any particular frequency band. Delta, alpha, and gamma are all valid contributors depending on the overall configuration.
\end{enumerate}

With these desiderata in mind, we construct $C(t)$ as a weighted sum of five components: mode entropy, participation ratio, phase coherence, entropy production rate, and criticality index. Within this framework, conscious experience corresponds not to activity in any particular region or frequency band, but to a global configuration of the harmonic field---an integrated pattern of interactions among spatial modes that collectively support differentiated yet unified dynamics. The substrate carries the signal, but the configuration carries the consciousness.

\subsection{Mode Power and Distribution}

The instantaneous power in mode $k$ is
\begin{equation}
P_k(t) = |a_k(t)|^2,
\end{equation}
where $a_k(t)$ is the (possibly complex) mode amplitude. The total power is $P_{\text{tot}}(t) = \sum_{k=1}^N P_k(t)$. We define the normalized mode distribution:
\begin{equation}
p_k(t) = \frac{P_k(t)}{\sum_{j=1}^N P_j(t)},
\end{equation}
satisfying $\sum_k p_k(t) = 1$ and $p_k(t) \geq 0$. This distribution encodes how energy is partitioned across the harmonic spectrum at time $t$.

\subsection{Mode Entropy}

The first component of the consciousness functional is the \emph{mode entropy}:
\begin{equation}
H_{\text{mode}}(t) = -\sum_{k=1}^N p_k(t) \log p_k(t),
\end{equation}
with the convention that $0 \log 0 = 0$. This is the Shannon entropy of the mode distribution, measuring the diversity or spread of energy across modes.

Mode entropy ranges from $H_{\text{mode}} = 0$ (all power in a single mode) to $H_{\text{max}} = \log N$ (uniform distribution). High mode entropy indicates that the field configuration is differentiated---no single mode dominates---while low entropy indicates concentration.

Empirically, increased mode entropy has been associated with richer conscious states. Psychedelics, for instance, increase the entropy of neural activity \cite{carhart2014entropic}, while anesthesia and deep sleep reduce it \cite{casali2013theoretically}.

\subsection{Participation Ratio}

A complementary measure of mode spread is the \emph{participation ratio}:
\begin{equation}
\text{PR}(t) = \frac{1}{\sum_{k=1}^N p_k(t)^2}.
\end{equation}
This quantity estimates the effective number of modes with significant power. If all power is in one mode, $\text{PR} = 1$; if power is uniformly distributed, $\text{PR} = N$.

While mode entropy and participation ratio are related, they weight the tails of the distribution differently. The participation ratio is less sensitive to small probabilities and provides a robust count of ``active'' modes. Both are included in $C(t)$ to capture different aspects of mode diversity.

\subsection{Phase Coherence}

The mode amplitudes can be written in polar form: $a_k(t) = r_k(t) e^{i\theta_k(t)}$, where $r_k(t) \geq 0$ is the magnitude and $\theta_k(t) \in [0, 2\pi)$ is the phase. The phases encode the timing relationships among modes---whether they oscillate in synchrony or with independent phases.

We define a \emph{phase coherence order parameter} analogous to the Kuramoto order parameter \cite{kuramoto1984chemical}:
\begin{equation}
R(t) = \left| \frac{1}{N} \sum_{k=1}^N e^{i\theta_k(t)} \right|.
\end{equation}
This quantity ranges from $R = 0$ (phases uniformly distributed, no coherence) to $R = 1$ (all phases aligned, maximal coherence).

The role of phase coherence in consciousness remains debated. Some theories emphasize global synchrony as a binding mechanism \cite{singer1999neuronal, fries2015rhythms}, while others highlight the importance of metastable, partially coherent states \cite{tognoli2014metastable}. We include $R(t)$ in the functional with moderate weight, reflecting the hypothesis that consciousness involves an intermediate regime---neither complete incoherence nor rigid lock-step synchrony. This interpretation aligns with recent work on oscillatory gating in mixed-selectivity coding, which demonstrates that phase alignment determines which neuronal features combine during readout \cite{padillacoreano2024mixed}. In the harmonic model, $R(t)$ captures exactly this: coherence across $\psi_k$ determines which combinations of spatial modes can produce high-dimensional mixed representations capable of supporting flexible cognition.

The relationship between $R(t)$ and cognitive function can be understood through three regimes:

\begin{itemize}
\item \textbf{Low coherence ($R \approx 0$):} When mode phases are uniformly distributed, the system exhibits maximal differentiation but minimal integration. Each mode evolves independently, and no global ``message'' can be read out. This corresponds to fragmented or disorganized states---neurologically, perhaps seizure activity or severe delirium.

\item \textbf{Intermediate coherence ($R \sim 0.3$--$0.7$):} Partial phase alignment enables selective binding: some mode combinations constructively interfere and reach downstream circuits, while others cancel. This metastable regime supports the flexible, context-dependent gating characteristic of wakeful cognition. Different phase configurations instantiate different ``gating patterns,'' allowing the same harmonic spectrum to support diverse cognitive operations depending on momentary phase relationships.

\item \textbf{High coherence ($R \approx 1$):} When all modes lock to the same phase, the system collapses to a single, rigid pattern. While globally integrated, such a state lacks the differentiation necessary for rich experience. This may correspond to hypersynchronous epileptiform activity or the stereotyped slow waves of deep anesthesia.
\end{itemize}

Consciousness, in this view, requires the ``Goldilocks zone'' of intermediate coherence---enough synchronization to bind distributed representations into unified experience, but enough phase diversity to maintain the high-dimensional, mixed-selective codes that underlie flexible thought \cite{padillacoreano2024mixed, miller2024neuron}.

\subsection{Entropy Production Rate}

A defining feature of conscious systems is their non-equilibrium character. The brain consumes substantial metabolic energy to maintain organized, low-entropy states far from thermodynamic equilibrium \cite{attwell2001energy}. The rate of entropy production quantifies the thermodynamic ``cost'' of maintaining such states.

For the linearized OU dynamics (Eq.~\ref{eq:OU}), the entropy production rate can be expressed in terms of the drift matrix $A$, noise covariance $\Sigma$, and state covariance $C(t)$ as \cite{tome2015entropy, seifert2012stochastic}:
\begin{equation}
\dot{S}(t) = \mathrm{Tr}\left( A \Sigma^{-1} A^T C(t) \right) - \mathrm{Tr}(A).
\end{equation}
At stationarity, $\dot{S} \geq 0$, with equality only at equilibrium (detailed balance). Higher $\dot{S}$ indicates the system is actively driven away from equilibrium by asymmetric dynamics.

Entropy production has been proposed as a marker of consciousness \cite{friston2010free}, with higher production associated with wakeful, engaged states and lower production with sleep or anesthesia. We include $\dot{S}(t)$ in the functional to capture this non-equilibrium signature.

\subsection{Criticality Index}

As discussed in Section~\ref{sec:modes-dynamics}, the brain appears to operate near a critical point---a phase transition between ordered and disordered regimes \cite{beggs2003neuronal, shew2013functional}. Near criticality, the system exhibits maximal dynamic range, sensitivity, and information transmission capacity.

We quantify proximity to criticality via the index
\begin{equation}
\kappa(t) = 1 - \frac{|\Re(\lambda_{\max})|}{\lambda_{\text{crit}}},
\end{equation}
where $\lambda_{\max}$ is the leading eigenvalue (most positive real part) of the linearized drift matrix $A$, and $\lambda_{\text{crit}}$ is a reference scale. When $|\Re(\lambda_{\max})| \ll \lambda_{\text{crit}}$, the system is near the stability boundary and $\kappa \approx 1$; when far from criticality, $\kappa \ll 1$ or negative.

Criticality supports the coexistence of local and global dynamics, transient synchronization, and flexible reconfiguration---properties associated with conscious processing \cite{tagliazucchi2012criticality}.

\subsection{The Combined Consciousness Functional}

We now combine the five components into a single scalar measure:
\begin{equation}
\label{eq:consciousness-functional}
\boxed{
C(t) = \alpha \frac{H_{\text{mode}}(t)}{H_{\max}} + \beta \frac{\text{PR}(t)}{\text{PR}_{\max}} + \gamma R(t) + \delta \frac{\dot{S}(t)}{\dot{S}_{\max}} + \varepsilon \kappa(t),
}
\end{equation}
where $\alpha, \beta, \gamma, \delta, \varepsilon > 0$ are positive weights, and the normalization factors $H_{\max} = \log N$, $\text{PR}_{\max} = N$, and $\dot{S}_{\max}$ (a characteristic scale for entropy production) ensure each term is dimensionless and of order unity.

Several remarks are in order:

\begin{enumerate}
\item \textbf{Interpretability.} Each term has a clear interpretation: $H_{\text{mode}}$ and $\text{PR}$ capture differentiation, $R$ captures integration/coherence, $\dot{S}$ captures non-equilibrium drive, and $\kappa$ captures criticality.

\item \textbf{Tunability.} The weights $\alpha, \beta, \gamma, \delta, \varepsilon$ are free parameters that can be adjusted based on empirical data or theoretical considerations. Different relative weights yield different predictions about which states are more or less conscious.

\item \textbf{Non-uniqueness.} We do not claim that Eq.~\eqref{eq:consciousness-functional} is the unique or correct consciousness functional. Rather, it is a principled, interpretable proposal consistent with leading theories. Alternative functionals are possible and may be compared empirically.

\item \textbf{Time dependence.} The functional is evaluated at each instant $t$, yielding a time series $C(t)$ that tracks fluctuations in consciousness. Averaging over time gives a mean level; variance reflects stability.
\end{enumerate}

\subsection{Relationship to Existing Measures}

The consciousness functional $C(t)$ relates to several existing complexity and consciousness measures:

\begin{itemize}
\item \textbf{Lempel-Ziv complexity and perturbational complexity index (PCI).} These measures quantify the compressibility or response diversity of neural signals \cite{casali2013theoretically}. High mode entropy and participation ratio correspond to low compressibility and high PCI.

\item \textbf{Integrated information ($\Phi$).} Tononi's integrated information theory defines consciousness as the irreducible information generated by a system \cite{tononi2004information}. While $C(t)$ does not compute $\Phi$ directly, the combination of entropy (differentiation) and coherence (integration) captures related intuitions.

\item \textbf{Global workspace measures.} Global workspace theory emphasizes the broadcasting of information across distant brain regions \cite{dehaene2001towards}. Low-index, globally coherent modes in our framework correspond to workspace activity; high $R(t)$ may index broadcasting.

\item \textbf{Criticality measures.} Neuronal avalanche analyses and power-law exponents have been used to assess criticality \cite{beggs2003neuronal}. The index $\kappa(t)$ provides a complementary, eigenvalue-based measure.
\end{itemize}

Thus, $C(t)$ synthesizes insights from multiple theoretical traditions into a unified, computationally tractable functional.

\subsection{Computation and Estimation}

In practice, $C(t)$ is computed as follows:

\begin{enumerate}
\item \textbf{Obtain mode amplitudes.} Given neural activity data $X(t)$ (e.g., source-reconstructed EEG), project onto the harmonic basis: $a_k(t) = \psi_k^T X(t)$.

\item \textbf{Compute power distribution.} Calculate $P_k(t) = |a_k(t)|^2$ and normalize to obtain $p_k(t)$.

\item \textbf{Evaluate mode entropy and participation ratio.} Apply the definitions above.

\item \textbf{Extract phases.} Use the Hilbert transform or analytic signal representation to obtain $\theta_k(t)$, then compute $R(t)$.

\item \textbf{Estimate entropy production.} This requires knowledge of the drift matrix $A$ and noise covariance $\Sigma$, which can be estimated from data via system identification or assumed from model parameters.

\item \textbf{Compute criticality index.} Evaluate the eigenvalues of $A$ and apply the definition of $\kappa(t)$.

\item \textbf{Combine terms.} Weight and sum to obtain $C(t)$.
\end{enumerate}

Windowed or time-averaged versions of these quantities can be used to smooth fluctuations and yield more stable estimates.

\subsection{Predictions and Testable Hypotheses}

The consciousness functional generates several predictions:

\begin{enumerate}
\item \textbf{Wakefulness vs. deep sleep.} Wakeful states should have higher $C(t)$ than deep NREM sleep, driven by higher mode entropy, participation ratio, and entropy production.

\item \textbf{Anesthesia.} General anesthesia should reduce $C(t)$ by decreasing entropy production and shifting dynamics away from criticality.

\item \textbf{Psychedelics.} Psychedelic states, associated with increased neural entropy \cite{carhart2014entropic}, should show elevated mode entropy and possibly $C(t)$.

\item \textbf{Dreaming.} REM sleep and dreaming should have intermediate $C(t)$---lower than wakefulness but higher than deep NREM---due to preserved mode diversity and criticality.

\item \textbf{Delta-rich consciousness.} Crucially, states with high delta power need not have low $C(t)$ if mode entropy, coherence, and criticality remain high. This is the delta paradox, to which we now turn.
\end{enumerate}

\subsection{Summary}

In this section we have defined a consciousness functional $C(t)$ that integrates five components:

\begin{itemize}
\item \textbf{Mode entropy} $H_{\text{mode}}(t)$: diversity of power distribution.
\item \textbf{Participation ratio} $\text{PR}(t)$: effective number of active modes.
\item \textbf{Phase coherence} $R(t)$: synchronization among modes.
\item \textbf{Entropy production rate} $\dot{S}(t)$: non-equilibrium drive.
\item \textbf{Criticality index} $\kappa(t)$: proximity to the edge of chaos.
\end{itemize}

The functional is mathematically explicit, computationally tractable, and connects to existing theories of consciousness. Its key virtue is that it depends on the \emph{configuration} of the harmonic field---the distribution, coherence, and dynamics of mode amplitudes---rather than on any single frequency band or regional activation. Fig.~\ref{fig:consciousness-components} illustrates how the five components and the combined $C(t)$ vary across brain states. This sets the stage for resolving the delta paradox.

\begin{figure}[t]
\centering
\includegraphics[width=0.9\linewidth]{figures/fig3_consciousness_components}
\caption{Components of the consciousness functional $C(t)$ across four brain states: wake, NREM unconscious, NREM dreaming, and anesthesia. (A--E) Each panel shows one normalized component: mode entropy $H_{\text{mode}}$, participation ratio PR, phase coherence $R$, entropy production $\dot{S}$, and criticality index $\kappa$. (F) The combined consciousness functional. Wake exhibits the highest $C(t)$; NREM unconscious and anesthesia the lowest; NREM dreaming shows intermediate values despite substantial low-mode power. The phase coherence component $R$ reflects phase-dependent mode-gating as described in mixed-selectivity research \cite{padillacoreano2024mixed}.}
\label{fig:consciousness-components}
\end{figure}
